
\documentclass[a4,10pt]{ctexart}

\usepackage{ctex}
\usepackage[utf8]{inputenc}
\usepackage{amsfonts,amsmath,amscd,amssymb,amsthm}
\usepackage{latexsym,bm}
\usepackage{cite}
\usepackage{mathtools,mathdots,graphicx,array}
\usepackage{fancyhdr}
\usepackage{lastpage}
\usepackage{color}
\usepackage{enumitem}
\usepackage{mpdoc}
\usepackage{diagbox}
\usepackage{xcolor,tcolorbox,tikz,tkz-tab,mdframed,tikz-cd}
\usepackage{framed}
\usepackage{verbatim}
\usepackage{extarrows}
\usepackage{fontspec}
\newcommand*{\dif}{\mathop{}\!\mathrm{d}}
\newcommand*{\arsinh}{\mathop{}\!\mathrm{arsinh}}
\newcommand*{\artanh}{\mathop{}\!\mathrm{artanh}}
\newcommand*{\arcosh}{\mathop{}\!\mathrm{arcosh}}
\newcommand*{\Li}{\mathop{}\!\textrm{Li}}


\begin{document}
\pagenumbering{roman}
\title{非线性分析笔记}
\author{math\_hyphen}
\date{2022年1月}
\maketitle
\tableofcontents
\newpage
\pagenumbering{arabic}
\newpage



\section{度理论}

\begin{lt}
    设$\phi\in{C'(\mathbb{R},\mathbb{R})}$ 满足对$\forall{x}\in{\mathbb{R}}$,有$\phi'(x)>0$,如果$\phi(1)=0$. 证明:
    \begin{equation}
      \left\{\begin{aligned}
      x^3-3xy^2+\phi(x) &= 1\\
      -y^3 + 3x^2y &=0
      \end{aligned}\right.  \label{a}
    \end{equation}
    上述方程组在$B(0,2)=\{(x,y)\in{\mathbb{R}^2}|x^2+y^2<4\}$上至少有3个解.
\end{lt}
  
\begin{proof}
    对$\forall(x,y)\in\overline{B(0,2)}$及$\forall{t\in[0,1]}$,定义
    \begin{equation}
      H(x,y,t) = (x^3-3xy^2+t\phi(x), -y^3+3x^2y)
    \end{equation}
    下面证明:$\forall{t}\in[0,1]$ 及 $(x,y)\in\partial{B(0,2)}$,有$H(x,y,t)\neq(1,0)$,否则, $\exists{t\in[0,1]}$及$(x,y)\in\partial{B(0,2)}$, s.t. $H(x,y,t)=(1,0)$,即
    \begin{equation}
      \left\{\begin{aligned}
      x^3 - 3xy^2 + t\phi(x) &= 1\\
      -y^3 + 3x^2y &= 0 \\
      x^2 + y^2 &= 4
      \end{aligned}\right. 
    \end{equation}
    可以计算出此时 $|t\phi(x)|\ge{7}$,与假设条件矛盾. 故 $\forall{t}\in[0,1]$及 $(x,y)\in{\partial{B(0,2)}}$, 有 $H(x,y,t)\neq(1,0)$.\\
    因此,利用Brouwer度的同伦不变性,有
    \begin{equation}
      \deg(H(\cdot, 1), B(0,2), (1,0)) = \deg(H(\cdot, 0), B(0,2), (1,0))=3
    \end{equation}
    而显然$x=1, y=0$是方程组 \eqref{a} 的解,由于
    \begin{equation}
      J_{H(\cdot,1)}(x,y) = \left|\begin{matrix}
      3x^2-3y^2+\phi'(x) & -6xy \\
      6xy & -3y^2+3x^2
      \end{matrix}\right|
    \end{equation}
    此时 $J_{H(\cdot, 1)}(1,0) = 9+3\phi'(1)>0$, 即 $x=1, y=0$是 $H(\cdot, 1)=(1,0)$在 $(1,0)$的一个充分小邻域 $B((1,0),\delta)$ 中的唯一解,并且 $\deg(H(\cdot, 1), B((1,0),\delta),(1,0))=1$.\\
    由切除性知, $\deg(H(\cdot, 1),B(0,2)\backslash\overline{B((1,0),\delta)},(1,0)) =2 \neq{0}$. (注意到 $y=0$时, $\phi(x)+x^3=1$, $x=1$是该方程的解,并且 $\phi'(x)+3x^1>0, \forall{x\in{\mathbb{R}}}$, 从而在$y=0$这条直线上, $x=1$是该方程的唯一解.)故,由Kronecker存在性定理知, $\exists(x,y)\in{B(0,2)\backslash\overline{B((1,0), \delta)}}$, s.t. $H(x,y,1)=(1,0)$, 并且 $y\neq{0}$. 注意到此时 $(x,-y)$也是方程 (\ref{a})的解. 因此,方程组在 $B(0,2)$ 中至少有三个解.
\end{proof}
\begin{lt}
  设 $\Omega\in\mathbb{R}^n$ 是有界趋于, $J\in\mathbb{R}$是区间, $f:J\times\Omega\to\mathbb{R}^n$是连续的,任意给定的 $(t_0, u_0)\in{J\times\Omega}$,讨论如下初值问题:
  \begin{equation}
    \left\{\begin{aligned}
    \frac{\dif{u}}{\dif{t}} &= f(t,u), t\in{J} \\
    u(t_0) & =u_0
    \end{aligned}\right. \label{b}
  \end{equation}
  如果 $f$ 关于 $u$ 是 Lipschitz 连续,问题 (\ref{b}) 存在唯一连续解;如果 $f$ 关于 $u$ 不是Lipschitz 连续,此时问题 (\ref{b}) 存在连续解,但是不一定唯一。
\end{lt}
\begin{lt}
  \begin{equation}
    \left\{\begin{aligned}
      &\frac{\dif{u}}{\dif{t}} = u^{\frac{2}{3}}, t\in{J} \\ 
      &u(t_0) = 0
    \end{aligned}\right.
  \end{equation}
\end{lt}

\begin{dl}{Krasnoselskii-Perov}{}
  设 $X$ 是Banach空间, $\Omega\in{X}$ 是有界开集, $f=I-F:\overline{\Omega}\to{X}$ 是全连续场, 假设下列条件成立:
  \begin{enumerate}
    \item 对 $\forall{\varepsilon}>0$, 存在全连续映射 $F_{\varepsilon}:\overline{\Omega}\to{X}$, s.t. $\forall{u}\in\overline{\Omega}$,有
      \begin{equation}
        \Vert{F_{\varepsilon}(u)-F(u)}\Vert < \varepsilon
      \end{equation}
    \item 当 $b\in{X}$且 $||b||<\varepsilon$, 方程 $u=F_{\varepsilon}(u)+b$ 至多有一个解.
  \end{enumerate}
  如果 $0\notin{f(\partial{\Omega})}$ 且 $\deg(f,\Omega, 0)\neq{0}$,则解集合 $S=\{u\in\Omega\left|f(u)=0\right.\}$ 是连通的.
\end{dl}

\begin{proof}
  $f$为全连续场得出: $f$固有且闭\\
  由 $\deg(f,\Omega, 0)\neq{0}$ 知 $S\neq{\emptyset}$, 且 $S=f^{-1}(\{0\})$是紧集. 如果 $S$不是连通的,那么 $S$可以表示为两个非空不交闭集的并,而紧集的任何闭子集都是紧集,所以可知此时 $S$可以表示为两个非空不交紧集的并, 从而,存在两个非空的开集 $U,V$ 满足 
  $$\overline{U}\cap\overline{V} = \emptyset, \quad U\cup{V}\supset{S}, \quad S\cap{U}\neq{\emptyset}, \quad S\cap{V}\neq\emptyset$$
  由 $S\cap{U}\neq\emptyset$ 知, $\exists{u}\in{S\cap{U}}$, 从而有 $f(u)=0$, 即 $u=F(u)$. 对 $\forall{\varepsilon}>0, \forall{v}\in\overline{V}$, 定义
  \begin{equation}
    f_{\varepsilon}:=v-F_{\varepsilon}(v)-(u-F_{\varepsilon}(u))
  \end{equation}
  起哄 $F_{\varepsilon}$由条件(1)给出. 容易验证此时, $u\in{U}$是 $f_{\varepsilon}(v)=0$在 $\Omega$中的唯一解.\\ 
  对 $\forall{v}\in\overline{V},\forall{t}\in[0,1]$, 令
  \begin{equation}
    H(t,v) = tf_{\varepsilon}(v) + (1-t)f(v)
  \end{equation}
  由于 $V$是 $S$的一个闭连通分支的开邻域及 $0\in{f(S)}$, 从而 $0\notin{f(\partial{V})}$. 由 $f$是闭的可知, $f(\partial{V})$也是闭的,于是,存在 $\alpha>0$, s.t. $\mathrm{dist}(0, f(\partial{V}))\ge\alpha>0$.\\ 
  对 $\forall{t}\in[0,1], v\in\partial{V}$, 有:
  \[
  \begin{aligned}
  \Vert{H(t,v)}\Vert &=  \Vert{f(v)+t(f_{\varepsilon}(v)-f(v))}\Vert \ge \Vert{f(v)}\Vert - t \Vert{f_{\varepsilon}(v)-f(v)}\Vert\\ 
  &\ge \Vert{f(v)}\Vert - \Vert{v-F_{\varepsilon}(v)-(F(u)-F_{\varepsilon}(u))-(v-F(v))}\Vert\\ 
  &\ge \Vert{f(v)}\Vert - 2 \Vert{F(v)-F_{v\varepsilon}(v)}\Vert\\ 
  &\ge \alpha - 2\varepsilon
  \end{aligned}  
  \]
  取 $0<\varepsilon<\frac{\alpha}{4}$, 则对 $\forall{v}\in\partial{V},\forall{t}\in[0,1]$,有
  $$\Vert{H(t,v)}\Vert\ge\frac{\alpha}{2}>0 \Rightarrow \forall{t}\in[0,1], 0\notin{H(t,\partial{V})}$$
  由 Leray-Schauder 度的同伦不变性可知, $\deg(H(t, \cdot), V, 0)$有意义且不依赖于 $t$,从而,有
  \begin{equation}
    \deg(f,V,0) = \deg(H(0,\cdot),V,0) = \deg(H(1,\cdot), V, 0) = \deg(f_{\varepsilon},V,0)
  \end{equation}
  由 $u\in{U}$是 $f_{\varepsilon}(v)=0$在 $\Omega$中的唯一解可知 $f_{\varepsilon}(v)=0$在 $V$中无解,得出 $\deg(f_{\varepsilon}, V, 0)=0$, 即有 $\deg(f,V,0)=0$.\\ 
  类似地, 证明 $\deg(f,U,0)=\deg(f_{\varepsilon},U,0)=0$. \\
  由 Leray-Schauder 度的切除性及区域可加性知:
  \[
  \begin{aligned}
  0 \neq \deg(f,\Omega,0) &= \deg(f,U\cup{V},0) \\ 
  & = \deg(f,U,0) + \deg(f,V,0)=0
  \end{aligned}  
  \]
  矛盾,故集合 $S$是连通的.
\end{proof}



\subsection{Leray-Schauder不动点定理}

\begin{lt}
设 $H=l^2, B=\{x\in{H}, \Vert{x}\Vert\le{1}\}$ 定义 $f:B\to{B}, f(x)=\left(\sqrt{1-\Vert{x}\Vert},x_1, x_2,\cdots\right)$, 则 $f$ 在 $B$ 上是一致连续的且 $f(B)\subset{B}$. 但是该映射并没有不动点.
\end{lt}

\begin{yl}
  设 $\Omega$ 是实线性赋范空间 $X$ 中的有界开集, $F:\overline{\Omega}\to{X}$ 全连续,对 $\forall{x}\in\overline{\Omega}$及 $t\in[0,1]$, 定义 $G(x,t)=tF(x)$, 则 $G:\overline{\Omega}\times[0,1]\to{X}$ 是两变元全连续的.
\end{yl}

\begin{proof}
  1. $\forall\{t_n\}\subset[0,1]$及 $\{x_n\}\subset\overline{\Omega}$ 是有界序列。由假设知, $\exists\{F(x_{n_j})\}\subset\{F(x_n)\}$ 及 $y\in{X}$, s.t. $F(x_{n-j})\to{y}$, $t_{n_j}\to{t_0}$.
可以得到 $t_{n_j}F(x_{n_j})\to{t_0y} \Rightarrow G(x_{n_j},t_{n_j})\to{t_0y}$
推出 $G$ 是紧映射.\\
2. if $t_n\to{t_0}, x_n\to{x_0}\Rightarrow t_nF(x_n)\to{t_0F(x_0)}$
\end{proof}


\begin{dl}{Leray-Schauder不动点}{}
  设 $\Omega\subset{X}$ 是实线性赋范空间 $X$ 中包含原点的有界开集, $F:\overline{\Omega}\to{X}$ 全连续映射且满足 Leray-Schauder 边界条件:对 
  $\forall{x}\in\partial{\Omega}$ 及 $\lambda>1$,有 $F(x)\neq{\lambda{x}}$,则 $F$ 在 $\overline{\Omega}$ 上至少有一个不动点.
\end{dl}
\begin{zy}
  \begin{enumerate}
    \item Leray-Schauder 边界条件的几何意义是:当$x\in\partial{\Omega}$时,$F(x)$不在$\{tx|t>1\}$上;
    \item Leray-Schauder 边界条件也可以写为 $\forall{x}\in\partial{\Omega}$及 $\mu\in[0,1)$,有$x\neq\mu{F(x)}$,即当$\mu\in[0,1)$时,如果方程$x=\mu{F(x)}$在$\overline{\Omega}$上有解,则$x$一定在 $\Omega$的内部.
  \end{enumerate}
\end{zy}
\begin{proof}
如果 $x\in{\partial\Omega}$, s.t. $F(x)=x$, 则结论显然成立.\\
不妨假设 $\forall{x}\in \partial\Omega$,有 $F(x)\neq{x}$. 对$\forall{x}\overline{\Omega}$及 $t\in[0,1]$, 令 $h_t(x)=x-tF(x)$, 则由假设知,\\
\begin{enumerate}
  \item 当 $o<t<1$时,对 $\forall{x}\in \partial\Omega$,有 $h_t(x)\neq{0}$
  \item 当 $t=0$时, 对$\forall{x}\in \partial\Omega$,有 $h_0(x)=x\neq{0}$
\end{enumerate}
利用 Leray-Schauder 度的同伦不变性知:
\[ 
  \begin{aligned}
    \deg({I-F},{\Omega}, {0}) &= \deg({h_1},{\Omega}, {0})\\
    &= \deg({h_0},{\Omega}, {0})\\
    &= \deg({I},{\Omega}, {0}) \\
    &= 1 \neq{0}    
  \end{aligned} 
\]
由 Kronecker 存在性定理知, $\exists{x}\in\Omega$ s.t. $x-F(x)=0$, i.e. $F(x)=x$
\end{proof}

\begin{tl}{}
  设 $\Omega\subset{X}$ 是实线性赋范空间 $X$ 中含有原点的有界开集, $F:\overline{\Omega}\to{X}$是全连续的. 如果下列条件之一成立:
  \begin{enumerate}
    \item (Rothe条件) 对 $\forall{x}\in\partial\Omega$,有 $\left\|{F(x)}\right\|\le \left\|{x}\right\|$
    \item (Altman条件) 对 $\forall{x}\in \partial\Omega$,有 $\left\|{x-F(x)}\right\|^2\ge \left\|{F(x)}\right\|^2 - \left\|{x}\right\|^2$
    \item (Krasnoselskii条件) 设 $X$是内积空间,对 $\forall{x}\in \partial\Omega$,有 $(F(x),x)\le \left\|{x}\right\|^2$
  \end{enumerate}
  则 $F$ 在 $\overline{\Omega}$ 上有不动点.
\end{tl}
\begin{proof}
  Leray-Schauder边界条件: $\forall{x}\in \partial\Omega$及 $\forall{\lambda}>1$,有 $F(x)\neq{\lambda{x}}$. 如果存在 $x\in \partial\Omega$ 及 $\lambda>1$, s.t. $F(x)=\lambda{x}$,则
  \begin{enumerate}
    \item $\left\|{F(x)}\right\|=\left\|{\lambda{x}}\right\|=\lambda \left\|{x}\right\|>\left\|{x}\right\|$,这与 $\left\|{F(x)}\right\|\le \left\|{x}\right\|$ 矛盾.
    \item 由边界条件推出与已知条件矛盾!
    \item $(F(x),x)=(\lambda{x},x)=\lambda \left\|{x}\right\|^2>\left\|{x}\right\|^2$,这与$(F(x),x)\le \left\|{x}\right\|^2$矛盾.
  \end{enumerate}
  综上所述,以上三个条件之一满足时, $F$在$\overline{\Omega}$上有不动点.
\end{proof}

\begin{tl}{}
  设 $\Omega\subset{X}$是实线性赋范空间 $X$中的有界闭凸集且包含内点. $F:\Omega\to{X}$是全连续的且 $F(\partial\Omega)\subset\Omega$,则 $F$在$\Omega$上有不动点.
\end{tl}
\begin{proof}
  由 $\Omega$ 有内点,不妨假设 $0$ 是$\Omega$ 的内点,否则,用 $\Omega-\{x_0\}$ 及$F(x+x_0)-x_0$ 来代替 $\Omega$ 和$F(x)$. 由于 $F(\partial\Omega)\subset\Omega$及 $\Omega$的凸性,故对$\forall{x}\in \partial\Omega$及 $\forall{\lambda}>1$,有
  \[
  \begin{aligned}
    \frac{\lambda-1}{\lambda}\cdot{0}+\frac{F(x)}{\lambda}\in\dot{\Omega} &\Rightarrow \text{对} \forall{x}\in \partial\Omega \text{及} \lambda>1, \text{有} \frac{F(x)}{\lambda}\in\dot{\Omega}\\
    &\Rightarrow \text{对} \forall{x}\in \partial\Omega\text{及} \lambda>1,\text{有} F(x)\neq{\lambda{x}}  
  \end{aligned}
  \]
  由 Leray-Schauder 不动点定理知, $F$ 在 $\Omega$ 上有不动点.
\end{proof}

\begin{tl}{}
  设 $\Omega$是实线性赋范空间 $X$中的有界开集, $F:\overline{\Omega}\to{X}$是全连续的. 如果存在$x_0\in\Omega$, s.t. 对$\forall{x}\in \partial\Omega$及$\alpha>1$有:
  \[F(x)-x_0\neq\alpha(x-x_0)\]
  则 $F$ 在 $\overline{\Omega}$上有不动点.
\end{tl}
\begin{proof}
  令 $\Omega'=\Omega-\{x_0\}\Rightarrow{x}\in\Omega'$ =nnnnkkhdefehfhhasjfasjfasfasjfgasjgfajfgajfaf=
\end{proof}
''''''

\begin{yd}{这是一个约定}{}
              
约定的内容.
\end{yd}

上述就是一些文本了. 使用的代码是
\begin{lstlisting}{language=latex}
\begin{yd}{这是一个约定}{}
              
约定的内容.
\end{yd}
\end{lstlisting}

\begin{zs}
        
这是一个注释
    
\end{zs}
    
\begin{xt}
        
这是一个习题.
    
\end{xt}
    
\begin{lt}
        
这是一个问题.
    
\end{lt}

\begin{yl}
        
这是一个引理.
    
\end{yl}

\begin{dl}{A}{}
        
这是一个定理.
    
\end{dl}
    
\begin{tl}{A}{}
        
这是一个推论.
    
\end{tl}

\begin{dy}{A}{}
        
这是一个定义.
    
\end{dy}

\begin{jl}{A}{}
        
这是一个结论.
    
\end{jl}

\begin{dl}{A}{}
        
这是一个命题.
    
\end{dl}
    
\begin{ti}{A}{}
        
这是一个题目.
    
\end{ti}

\begin{cx}{A}{}
        
这是一个猜想.
    
\end{cx}

\begin{zy}
        
这是注意.
    
\end{zy}

\begin{ts}
        
这是一点提示.
    
\end{ts}

\begin{lt}
        
这是一个例题.
    
\end{lt}

\begin{proof}
你还可以加一点证明. 
\end{proof}

我们注意到, 所有的数学公式将自动转换成行间公式的大小, 比如${1\over 2^k}$, $\sum_{i=0}^{998244353}i$. 

\section{起源与未来的修改计划}

起源与hkmod的模板, 添加了一些常用的标志词. 可以在mpdoc.sty里面进行更改, 相信根据注释你也会. 

使用愉快! 
    
    
    
    
   

\end{document}
